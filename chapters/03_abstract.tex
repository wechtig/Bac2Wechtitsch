%**********************************************************************

%---------------------------------------------------
% NOTE:
% English version of the abstract is always required 
% (even for German BA/MAs)
%---------------------------------------------------

% right side/flush
\chapterend

\begin{titlepage}

\begin{otherlanguage}{english} 

\begin{abstract} % Abstract

Your text here\ldots
Write the abstract in English and in German, called \emph{Zusammenfassung}.
Describe in about 250 to 350 words the problem, the innovation, the method, the results and implications.

\end{abstract}

\end{otherlanguage}


\end{titlepage}


%---------------------------------------------------
% NOTE:
% German version of the abstract "Zusammenfassung"
% is required (also for German BA/MAs, compare "actions" @ FHJ)
%---------------------------------------------------

\begin{titlepage}

\begin{otherlanguage}{german}

\begin{abstract}  % Zusammenfassung

Im Bereich der Software-Entwicklung in Teams gibt es mehrere verschiedene Herausforderungen, hierzu zählen zum Beispiel Punkte wie das Projektumfeld oder die sozialen Aspekte im Team. Aber auch Herausforderungen in der technischen Umsetzung sind von großer Bedeutung, unter anderem die Code Qualität, da sie in Projekten oft nicht genug gewichtet oder zu generell erfasst wird. Das Ziel ist es nun, aufbauend auf der Bachelorarbeit 1 \textit{Code Qualität und der Einsatz der Statischen Code Analyse für weitere Auswertungen und Präsentationen der Daten},  eine Möglichkeiten zur Teamunterstützung zu integrieren und folgend die einzelnen Teammitglieder besser zu unterstützen und damit die Kenntnisse dauerhaft zu steigern. Eine Herausforderung ist hierbei die fehlende Zuordnung der auftretenden Meldungen zu den jeweiligen Entwicklerinnen oder Entwicklern. \\\\
Dazu werden verschiedene Funktionen und Applikationen entwickelt. In einer Webapplikation sollen genauere Informationen zu den Meldungen angezeigt werden. Die Meldungen basieren auf den Ergebnissen der Plugins für die Statischen Code Analyse, die in den Projekten eingesetzt werden können. Eine mobile Android-Lösung soll die Verwendung und die Benutzung erleichtern. Dazu sollen die einzelnen Funktionen auf ein Projekt und ein dazugehöriges Team fokussiert werden. Eine Integration mit einer Versionsverwaltung soll die einzelnen Analysen und Meldungen genauer gestalten. \\\\
Die Evaluierung und die Ergebnisse der Implementierung zeigen die Möglichkeit auf, Aspekte zur Verbesserung der Code Qualität in einer Webapplikation für ein Team zu integrieren. Verschiedene Hilfsmittel wie automatische Reports oder mobile Applikationen können den Effekt dabei verbessern. Ebenso bieten sich Elemente der \textit{Gamification} an, sodass Entwicklerinnen und Entwickler zusätzlich mit den Aspekten der Code Qualität interagieren. 
\end{abstract}

\end{otherlanguage}

\end{titlepage}

%**********************************************************************
