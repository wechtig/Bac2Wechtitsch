%**********************************************************************

%---------------------------------------------------
% NOTE:
% English version of the abstract is always required 
% (even for German BA/MAs)
%---------------------------------------------------

% right side/flush
\chapterend

\begin{titlepage}

\begin{otherlanguage}{english} 

\begin{abstract} % Abstract

Software development in teams is followed by many different challenges, including points such as the interaction and the relationship with project managers, other stakeholders, or the social aspects in the team.
On the other hand the problems and challenges with the technical aspect are also of great importance, for example the code quality. Code quality is often an underrated part in the software developement. 
Therefore the main goal of this thesis and the associated implementions is to use the static code analysis for further evaluations and presentations of the collected data. 
This information should be used in teams, so the team support is also a big part. With this information and implementations, the knowledge of the developers should be increased permanently. The challenge is the lack of assignment for the messages that occur to the respective developers. \\\\
Therefore, diffrent functions and applications were developed. First a Webapplication, where more detailed information is provided. This information and the generated charts are based on 
the results of the static code analysis. A plugin imports this data into the database. In addition to this Webapplication, a mobile android application shows the most important information together with charts
and should make it easier and faster to use this collected data. This functions and applications are based on project with an associated team. The Webapplication interacts with a version management tool to create more personalized analysis. \\\\
The results show the possibility to improve the code quality and the knowledge of the team. Various functionalities such as automatic reports or mobile applications can accomplish that.
The gamification is also a possibility, to bring developers and code quality aspects further together. 


\end{abstract}

\end{otherlanguage}


\end{titlepage}


%---------------------------------------------------
% NOTE:
% German version of the abstract "Zusammenfassung"
% is required (also for German BA/MAs, compare "actions" @ FHJ)
%---------------------------------------------------

\begin{titlepage}

\begin{otherlanguage}{german}

\begin{abstract}  % Zusammenfassung

Im Bereich der Software-Entwicklung in Teams gibt es mehrere verschiedene Herausforderungen. Hierzu zählen zum Beispiel Punkte wie das Projektumfeld oder die sozialen Aspekte im Team. Aber auch Herausforderungen in der technischen Umsetzung sind von großer Bedeutung, unter anderem die Codequalität, da sie in Projekten oft nicht genug gewichtet oder zu generell erfasst wird. Das Ziel ist es nun, aufbauend auf Lösungen für einzelne Entwicklerinnen und Entwickler, Möglichkeiten zur Teamunterstützung zu integrieren und folgend die einzelnen Teammitglieder besser zu unterstützen und damit die Kenntnisse dauerhaft zu steigern ~\parencite{wechtitschCodeQuality}. Eine Herausforderung ist hierbei unter anderem die fehlende Zuordnung der auftretenden Meldungen zu den jeweiligen Entwicklerinnen oder Entwicklern. \\\\
Dazu werden verschiedene Funktionen und Applikationen entwickelt. In einer Webapplikation sollen genauere Informationen zu den Meldungen angezeigt werden. Die Meldungen basieren auf den Ergebnissen der Plugins für die Statischen Code Analyse, die in den Projekten eingesetzt werden können. Eine mobile Android-Lösung soll die Verwendung und die Benutzung erleichtern. Um diese Applikationen entwickeln zu können, sollen die einzelnen Funktionen auf ein Projekt und ein dazugehöriges Team fokussiert werden. Eine Integration mit einer Versionsverwaltung soll die einzelnen Analysen und Meldungen genauer gestalten. \\\\
Die Evaluierung und die Ergebnisse der Implementierung zeigen die Möglichkeit auf, Aspekte zur Verbesserung der Code Qualität in einer Webapplikation für ein Team zu integrieren. Verschiedene Hilfsmittel wie automatische Reports oder mobile Applikationen können den Effekt dabei verbessern. Ebenso bieten sich Elemente der \textit{Gamification} an, sodass Entwicklerinnen und Entwickler zusätzlich mit den Aspekten der Code Qualität interagieren. 
\end{abstract}

\end{otherlanguage}

\end{titlepage}

%**********************************************************************
