%%%%%%%%%%%%%%%%%%%%%%%%%%%%%%%%%%%%%%%%%%%%%%%%%%%%%%%%%%%%%%%%%%%%%%%%%%%%%
\chapter{Konzept}
\label{chap:concept}
%%%%%%%%%%%%%%%%%%%%%%%%%%%%%%%%%%%%%%%%%%%%%%%%%%%%%%%%%%%%%%%%%%%%%%%%%%%%%
\chapterstart

Die bestehende Applikation zur Visualisierung und Auflistung von den Ergebnissen der Statischen-Code-Analyse wird als Basis genommen. Um Unterstützungen für das Team bereitstellen zu können, wird die bestehende Applikation und erweitert und weiteres eine Lösung für mobile Geräte entwickelt. Die Daten für das erweiterte Backend und die mobile Applikation kommen aus einer Datenbank, welche mit einem Plugin für verschiedene Projekte befüllt wird.

\subsection{Bestehende Datenbank und Applikationen}
Das Plugin wird aus der Bachelorarbeit 1 übernommen, die Datenbank um mehrere Tabellen erweitert. Zu der bestehenden Web-Applikation werden neue Features entwickelt. 

\subsection{Funktionalitäten}
Um eine Teamfunktion integrieren zu können und damit die Code Qualität steigern zu können, muss eine Userverwaltung hinzugefügt werden. User müssen sich registrieren und anmelden können. Ebenso muss eine Team- und Projektverwaltung implementiert werden. Hierbei muss auch die Rechteverwaltung beachtet werden, da zum Beispiel nur Teammitglieder Projektdaten sehen dürfen und nur ausgewählte Benutzerinnen und Benutzer auf ein Projekt zugriff haben dürfen. Um die einzelnen Fehler und Bugs zuordnen zu können, muss eine Interaktion mit einem Programm für die Versionsverwaltung implementiert werden. Über diese Versionsverwaltung kann unter anderem festgestellt werden, welche Teammitglieder welche Bugs implementiert haben. 
\subsection{Mobile Applikation}

\subsection{Infrastruktur und Interaktion der Teilprojekte}
 
\chapterend