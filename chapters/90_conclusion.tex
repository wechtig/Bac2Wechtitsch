%%%%%%%%%%%%%%%%%%%%%%%%%%%%%%%%%%%%%%%%%%%%%%%%%%%%%%%%%%%%%%%%%%%%%%%%%%%%%
\chapter{Fazit und Ausblick}
\label{Fazit und Ausblick}
%%%%%%%%%%%%%%%%%%%%%%%%%%%%%%%%%%%%%%%%%%%%%%%%%%%%%%%%%%%%%%%%%%%%%%%%%%%%%
\chapterstart

Im Rahmen der Arbeit wurden verschiedene Lösungsansätze  entwickelt, die das Team bei der Entwicklung unterstützen sollen. Die Fokussierung lag hierbei auf der Code Qualität. Das Ziel war es auch, die Funktionalitäten individuell zu implementieren, um sie bei verschiedenen Projekten einsetzen zu können. \\\\
Die neuen Funktionalitäten und die Evaluierungen zeigen die Möglichkeit auf, die Kenntnisse des Projektteams über die Code Qualität im Projekt zu steigern und die Code Qualität im Projekt zu verbessern. Die Funktionalitäten teilen sich hierbei in zwei Bereiche: In eine Webapplikation und in eine mobile Lösung für Android-Geräte.\\\\
Die hierbei entwickelten Funktionen sind hierbei aber nur die Basis und bieten den Ausgangspunkt für weitere Entwicklungen an. Mit einer Integration in Entwicklungsumgebungen können beispielsweise Fehler schneller gefunden werden. Weiteres könnten weitere Teamfunktionalitäten, wie die Implementierung eines Punktesystem für Benutzerinnen und Benutzer mit der Basis der Fehlerausbesserung neue Anreize für das Projektteam schaffen und damit die Code Qualität in Projekten steigern.
\chapterend



%%
% Hints by Daniela Holzer 2017
% "instructions for composing degree papers.pdf"
%
% Formal Guidelines
%   Diploma thesis: 17 000 words ± 10% (excluding appendix)
%   Each of the two Bachelor papers: 10 000 words ± 10% (excluding appendix)
%% 