%%%%%%%%%%%%%%%%%%%%%%%%%%%%%%%%%%%%%%%%%%%%%%%%%%%%%%%%%%%%%%%%%%%%%%%%%%%%%
\chapter{Stand der Technik und Konzepte}
\chapterstart

\label{chap:related}
\section{Herausforderungen in der Entwicklung in Teams (Kultur und menschliche Aspekte)}
\subsection{Agile Teams}
Eine der großen Herausforderungen in der heutigen Softwareentwicklung ist die Entwicklung in Teams. Soziale Herausforderungen, Probleme im Prozess, Schwierigkeiten im Unternehmen und andere Aspekte sind ebenso wichtig in der Entwicklung in Teams, wie die technischen Aspekte. In dieser Arbeit werden Teams als Scrum-Teams im agilen Prozess/Unternehmen betrachtet: Die Teams können sich selbst organisieren und entwickeln sich selbst im Rahmen von Feedback, Erfahrung und Meetings weiter ~\parencite{cohn2003introducing}. Dieses Team arbeitet im Rahmen eines Prinzipes, welches ist ~\parencite{beck2001agile}
\begin{itemize} 
\item Individuen und Interaktionen mehr als Prozesse und Werkzeuge
\item Funktionierende Software mehr als umfassende Dokumentation
\item Zusammenarbeit mit dem Kunden mehr als Vertragsverhandlung
\item Reagieren auf Veränderung mehr als das Befolgen eines Plans
\end{itemize}
Das Team kann natürlich nur funktionieren, wenn es von der Organisation unterstützt wird. Dazu zählen folgende Punkte: Das Team kann miteinander interagieren und sich auf eine bestimmte Aufgabe konzentrieren. Zweitens sollte das Ziel eine sinnvolle Herausforderung sein. Die Teamstruktur ist an die Aufgabe angepasst und die Rahmenbedingungen unterstützen das Team, sodass sich das Team auf das Ziel konzentrieren kann. Zuletzt sollte das Team in einen angemessenen Zyklus bzw. Prozess eingebunden werden. ~\parencite{boos2017fuhrung}
\subsection{Teamkultur}
Unter einer Teamkultur versteht man das Pflegen von Regeln, Richtlinien und Verhaltensweisen ~\parencite{teamculture}. Eine gute Teamkultur ist die Basis für ein funktionierendes Team. Darunter fallen Bereiche, wie gemeine Vorstellungen und Ziele mit dem Projekt, Motivation, Umgang mit Fehlern, ein aufrichtiges Feedback und eine Gleichberechtigung untereinander im Team ~\parencite{teamcultureGrolman}. 
\subsection{Kommunikation}
Die Kommunikation ist eine der wichtigsten Einzelfaktoren für den Erfolg eines Projekts. Die Kommunikation sollte direkt, persönlich und offen sein. Die Kommunikation sollte hierbei auf mehreren verschiedenen Arten erfolgen, um möglichst effektiv zu sein. Die Kommunikation in größeren Teams ist eine größere Herausforderungen, da alle Personen gleich eingebunden werden sollen. Eine zu undirekte oder eine fehlende Kommunikation kann zu Vertrauns- und Motivationsverlust führen. Auch regelmäßige Meetings können die Kommunikation stärken. Großraumbüros können die Kommunikation und die Zusammenarbeit verbessern, dürfen aber nicht zu viele Personen umfassen, da sonst ein konzentriertes Arbeiten schwerer möglich ist. Besonders bei verteilten Teammitgliedern sollten Informationen rasch und zentral zugänglich sein, wie in Wikis und eigenen Nachrichtenkanälen. 
\subsection{Verteilte Teammitglieder}
Die Entwicklung in verteilten Teams stellt eine besondere Herausforderung da: Die Kommunikation ist schwieriger und das Teamgefühl ist weniger stark ausgeprägt. Dazu können technische Probleme auftreten. Auch ist der Security Aufwand größer ~\parencite{sutherland2007distributed}. Agile Teams haben hier einen Vorteil, da sie sich selbst organisieren können und viele Meetings in ihrem Prozess beinhalten, die besonders bei verteilten Teammitgliedern regelmäßig abgehalten werden sollen. Ebenso kann mittels Screen-Sharing, Pair-Programming oder festgelegten Fristen und Terminen die Produktivität und Zufriedenheit in verteilten Teams gesichert werden. 
\subsection{Rollen und Verantwortungen}
In agilen Teams sollte das gesamte notwendige Wissen für das Projekt im Team vorhanden sein. Der Wissensstand und die Kenntnisse können bzw. sollen sich daher unterscheiden, man spricht von \textit{cross-functional} Teams.  Ein Mitglied des Teams (Scrum Master) leitet das Team und vertritt es nach außen. Die Rollen sind daher meist klar, wenn nicht sollten sie klar festgelegt und definiert werden. Ebenso ist es bei der Verantwortung: In agilen Teams trägt jedes Teammitglied die Verantwortung selbst, wenn nicht, muss auch dies definiert werden. Das Team kann bei Vertrauen, Motivation und Selbstorganisation die Verantwortung gut tragen.
\subsection{Problem- und Konfliktbehandlung}
Der erste Schritt in der Konfliktlösung ist es herauszufinden, ob ein Konflikt oder ein Problem existiert. Dies kann zum Beispiel durch Schuldzuweisungen, mangelndes/schlechtes Feedback, geringere Leistungen oder fehlende Kommunikation und Motivation festgestellt werden. Konflikte und Probleme sollten dann direkt angesprochen werden. Gibt es Probleme bei größeren Aufgaben, besteht die Möglichkeit die Aufgaben neu zu verteilen oder outsourcen. Bei größeren Problemen ist auch ein Austausch von Teammitgliedern möglich. Wegen des großen Aufwands einer Einschulung und Einfindung eines neuen Mitglieds in das Team, sollte dieser Schritt möglichst vermieden werden. 
\section{Herausforderungen in der Entwicklung in Teams (Aspekte im Prozess)}
\subsection{Entwicklungsprozess: Iteration und Release}
In agilen Modellen wird in Iterationen entwickelt. Diese Zyklen dauern zwei bis vier Wochen. Je größer das Team und das Projekt, desto kürzer sollen die Zyklen sein, da sonst zu große und unkontrollierte Änderungen geschehen. Am Ende der Iteration gibt es ein Ergebnis, welches dem Kunden mit einem Release geliefert wird. So entsteht ein kontinuierlicher Fortschritt. Ein wichtiger Punkt hierbei ist die \textit{Definition of Done}: Wann ist etwas fertig? Nach der Abnahme, nach dem Testen oder nach der Entwicklung? Die Antwort auf diese Frage muss im Team und mit dem Kunden definiert werden. Ein anderer wichtiger Punkt bei der Abnahme und beim Release ist die Infrastruktur, die eine schnelle und problemfreie Integration unterstützt. ~\parencite{ecksteinTeams}
\subsection{Planen und Schätzen}
Die einzelnen zu entwickelnden Features werden zu Beginn der Iteration definiert und geschätzt. Die einzelnen Features sollen hierbei klein gehalten werden. Vor den Schätzen muss zum Beispiel auch definiert werden, ob die angegebene Zeit auch das Testen und verschiedene Reviews beinhaltet. Das Schätze sollten zusammen im Team erfolgen. Die einzelnen Tasks sollen daher zusammen diskutiert werden, auch wenn andere Teammitglieder mit dem einzelnen Task bzw. Feature weniger vertraut sind. Ist das Team neu gestaltet, kann es zu Beginn der Entwicklung zu starken Abweichungen mit der vorgenommen Zeit kommen. Vor dem Schätzen und Planen eines neuen Tasks, ist es auch wichtig, eine gemeinsame Basis zu schaffen um Unklarheiten zu klären. 
\subsection{Integration}
Bei der Integration werden die verschiedenen entwickelten Elemente in das System hinzugefügt. Die Integration ist ein großer Aufwand und ein wichtiger Teil des Prozesses. Wegen des großen Aufwands sollte man nicht zu oft integrieren, was im Gegensatz zu den kurzen Iterationen steht. Hier sollte ein Mittelweg gefunden werden. Bei mehreren Teams können auch eigene Teammitglieder das Integrieren übernehmen: Die Integrationsteams. Bei der Integration unterscheidet man zwischen synchroner und asynchroner Integration: Bei der synchronen Integration wird ein Teil nach dem anderen integriert, was weniger fehler- und konfliktanfällig, dafür aber zeitaufwendig ist. Bei der asynchronen Integration werden dagegen alle Teile auf einmal integriert. ~\parencite[vgl. 106-110]{ecksteinTeams}
\section{Code Qualität und die Entwicklung in Teams}
\subsection{Allgemeines}
Die Qualitätskontrolle soll besonders in agilen Projekte zusammen mit dem Kunden erfolgen. So können Probleme früh erkannt und ausgebessert werden. Im Idealfall ist die Qualitätskontrolle ein vollwertiger Teil des Teams. So kann auch genug Zeit für die Qualität verwendet werden. Ist die Qualitätskontrolle kein eigener Teil des Teams, so besteht die Gefahr, dass die Qualität vernachlässigt wird. Häufig werden die Qualitätskontrollen dann auch erst am Ende einer Iteration durchgeführt ~\parencite[vgl. 180-182]{ecksteinTeams}. Außerdem besteht die Gefahr, dass Führungskräfte in Unternehmen die Code Qualität als unwesentlich betrachten und wenig Aufwand und Ressourcen dafür verwenden wollen. \\
Bezüglich der Code Qualität und die Entwicklung in Teams, gibt es verschiedene Punkte um die Code Qualität sicher zu stellen ~\parencite{verwijsTeams}
\begin{itemize} 
\item Code Qualität definieren \\
Zu Beginn der Code Qualität sollte der Begriff genauer definiert werden. Hierzu können unter anderem mehrere Punkte zählen: Wie viel Prozent macht die Test Coverage aus? Wie viele Fehler und Warnungen darf ein Analyse-Tool werfen? Wie einfach ist der Code zu verstehen? 
\item Code Qualität unterstützen \\
Code Qualität ist ein wichtiger Aspekt in der Software Entwicklung. Deshalb sollte das Team genug Zeit dafür in Anspruch nehmen. Ebenso muss die Projektleitung und das Management Ressourcen für die Code Qualität freigeben.
\item Reviews \\
In Code Reviews werden die Teilergebnisse in einer Iteration betrachtet und überprüft. Dies geschieht durch die anderen Teammitglieder, die den Code korregieren und auf Fehler untersuchen, es können aber auch Patterns und Architektur-Entwürfe überprüft werden. Code Reviews sollten direkt in den Entwicklungsprozess eingebunden werden. So können sie beispielsweise in Form von Pull Requests in den Git-Workflow integriert werden. Bei einem Code Review werden daher sowohl der Code als auch die Kenntnisse des Reviewers verbessert. 
In Code Reviews werden meistens funktionelle Probleme bei der Evolvabilität festgestellt. ~\parencite{mantyla2008types}. Unter Evolvabilität versteht man die Fähigkeit einer Software weiterentwickelt zu werden und neuen Anforderungen gerecht zu werden. Besonders für neue Entwicklerinnen und Entwickler sind Code Reviews ein gute Möglichkeit, um mit dem Code des Projekts vertraut zu werden.
\item Andere Entwicklerinnen und Entwickler \\
Auch Entwicklerinnen und Entwickler die nicht direkt am Projekt beteiligt sind, könnten regelmäßig den Code reviewen und so wesentlich zur Code Qualität beitragen. Sie können neue Impulse und Erfahrungen aus anderen Projekten einbringen und haben möglicherweise einen differenzierteren Blick auf das Projekt und den Code. 
\item Evaluierungen und regelmäßige Überprüfungen, auch durch externe Audits. 
\end{itemize}
\subsection{Refactoring}
Beim Refactoring wird der Code verbessert, ohne dass das Verhalten des Programms verändert wird. Dadurch wird schlechter und fehleranfälliger Code ausgebessert und in sauberen Code umgearbeitet. Der Code wird vor dem Refactoring auf verschiedene Aspekte der Code Qualität geprüft, um herauszufinden welche Teile ausgebessert werden sollen: Testbarkeit, Wartbarkeit, Erweiterbarkeit, Einfachheit, Sicherheit, usw.  ~\parencite{fowler2018refactoring} 
\subsection{Probleme beim Refactoring}
Beim Refactoring können mehrere verschiedene Probleme auftreten, die das Refactoring erschweren. Beispielsweise kann die unkontrollierte Anbindung von vielen Interfaces, den Aufwand des Refactorings einer Klasse stark steigern. Auch das häufige Umbenennen von Methoden, Klassen oder Variablen kann aus zwei Gründen problematisch werden: Erstens führt es zu einer Kette von Nachfolgeänderungen und zweitens können auch Probleme auftreten, wenn mit dem Team die Änderungen nicht kommuniziert werden, da zum Beispiel die Methoden bzw. Klassen nicht mehr gefunden werden oder die Methodensignatur sich ändert. Auch fehlende Unit-Tests und häufige Änderungen der Architektur und des Designs sind ein Probleme für das Team und die Qualität der Software. ~\parencite{khanambarriers} \\
Um das Refactoring effektiver zu gestalten, sollten die Ziele und die verschiedenen Änderungen im Team besprochen werden. Ebenso kann auch beim Refactoring Pair bzw. Mob-Programming eingesetzt werden, da dadurch der Lerneffekt und das Wissen der Teammitglieder über den Code gesteigert werden kann, diese Methode ist aber aufgrund des Zeitaufwands nicht sehr verbreitet. 
\subsection{Maintainability}
Unter Maintainability bzw. Wartbarkeit versteht man wie einfache Änderungen und Erweiterungen in der Software durchgeführt werden können. Aber auch Bug Fixing und das Ausbessern von Security-Problemen kann durch eine bessere Maintainability vereinfacht werden. Die Kriterien der Maintainability können hierbei sein: Testbarkeit, Einfachheit und Aufwand zur Veränderung ~\parencite{kukrejaMaintainability}. Um die Maintainability zu steigern sollten kleine Methoden und Klassen geschrieben werden, Code nicht doppelt geschrieben werden, Tests implementiert werden, Architektur lose gekoppelt sein und die Größe des Codes minimal sein. Die Kosten für die Wartbarkeit werden in drei Teile unterschieden: Korrektive Maßnahmen (Beinhaltet das Ausbessern von Fehlern im Code für die funktionalen Anforderungen), Adaptive Maßnahmen (Beinhaltet die Vorbereitung des Codes für die Implementierung von neuen Features) und Design Maßnahmen (Beinhaltet das Ausbessern des Design oder die Erhöhung der Performance. Wird oft durch Reviews oder Audits ausgelöst) ~\parencite{cheaitoMaintainability}.
\subsubsection{Metriken für die Maintainability}
Die Metriken für die Wartbarkeit können unterschiedlich sein und sollten daher im Team definiert werden. So kann bei der Entwicklung im Team auf bestimmte Metriken mehr geachtet  werden. Die Metriken hängen aber auch von der Art des Projekts (Programmiersprachen, Größe, Einsatzgebiet) stark ab. \\
Einige herkömmliche Metriken können unter anderem sein:

\begin{itemize} 
\item Ausprägung der Abhängigkeit und der Koppelung zwischen den Modulen 
\item Kohäsion (Abbildung der Funktionalität durch einen bestimmten Codeteil)
\item Komplexität im Design \\
Als wichtigste Design-Faktoren gelten Accessibility, Test points und Controls. ~\parencite{dhillonMaintainability}.
\item McCabe-Metrik (Anzahl der Pfade durch ein System. Je mehr Pfade es durch ein System gibt, desto komplexer ist das System ~\parencite{curtis1979measuring}.)
\item Anzahl der Kommentare
\item Länge der Variablennamen
\item Größe der Klassen und Methoden
\end{itemize}

Der hier beschriebene Stand der Technik und die verschiedenen Aspekte der Teamentwicklung und der Software Qualität geben die Basis für das Konzept und die Entwicklung der Arbeit.
\chapterend