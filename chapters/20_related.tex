%%%%%%%%%%%%%%%%%%%%%%%%%%%%%%%%%%%%%%%%%%%%%%%%%%%%%%%%%%%%%%%%%%%%%%%%%%%%%
\chapter{Stand der Technik und Konzepte}
\chapterstart

\label{chap:related}
\section{Herausforderungen in der Entwicklung in Teams (nicht-technische Aspekte)}
\subsection{Agile Teams}
Einer der großen Herausforderungen in der heutigen Softwareentwicklung ist die Entwicklung in Teams. Soziale Herausforderungen, Probleme im Prozess, Schwierigkeiten im Unternehmen und andere Aspekte sind ebenso wichtig in der Entwicklung in Teams, wie die technisches Aspekte. In dieser Arbeit werden Teams als Scrum-Teams im agilen Prozess/Unternehmen betrachtet: Die Teams können sich selbst organisieren und entwickeln sich selbst im Rahmen von Feedback, Erfahrung und Meetings weiter \ref{cohn2003introducing}. Dieses Team arbeitet im Rahmen eines Prinzipes, welches ist \ref{beck2001agile}
\begin{itemize} 
\item Individuen und Interaktionen mehr als Prozesse und Werkzeuge
\item Funktionierende Software mehr als umfassende Dokumentation
\item Zusammenarbeit mit dem Kunden mehr als Vertragsverhandlung
\item Reagieren auf Veränderung mehr als das Befolgen eines Plans
\end{itemize}
\subsection{Teamkultur}
Unter einer Teamkultur versteht man das pflegen von Regeln, Richtlinien und Verhaltensweisen \ref{teamculture}. Eine gute Teamkultur ist die Basis für eine funktionierendes Team. Darunter fallen Bereiche wie, wie gemeine Vorstellungen und Ziele mit dem Projekt, Motivation, Umgang mit Fehlern, ein aufrichtiges Feedback und eine Gleichberechtigung untereinander im Team. \ref{teamcultureGrolman}. 
\subsection{Kommunikation}
Die Kommunikation ist einer der wichtigsten Einzelfaktoren für den Erfolg eines Projekts. Die Kommunikation sollte direkt, persönlich und offen sein. Die Kommunikation sollte hierbei auf mehreren verschiedenen Arten erfolgen, um möglichst effektiv zu sein. Die Kommunikation in größeren Teams ist eine größere Herausforderungen, da alle Personen gleich eingebunden werden sollen. Eine zu undirekte oder eine fehlende Kommunikation kann zu Vertrauns- und Motivationsverlust führen. Auch regelmäßige Meetings können die Kommunikation stärken.
\subsection{Verteilte Teammitglieder}
Die Entwicklung in verteilten Teams stellt eine besondere Herausforderung da: Die Kommunikation ist schwieriger und das Teamgefühl ist weniger stark ausgeprägt. Dazu können technische Probleme auftreten. Auch ist der Security Aufwand größer. \ref{sutherland2007distributed} Agile Teams haben hier einen Vorteil, da sie sich selbst organisieren können und viele Meetings in ihrem Prozess beinhalten, die besonders bei verteilten Teammitgliedern regelmäßig abgehalten werden sollen. Ebenso kann mittels Screen-Sharing, Pair-Programming oder festgelegten Fristen und Terminen die Produktivität und Zufriedenheit in verteilten Teams gesichert werden.
\subsection{Rollen und Verantwortungen}
In agilen Teams sollte der gesamte notwendige Wissen für das Projekt im Team vorhanden sein. Der Wissenstand und die Kenntnisse können bzw. sollen sich daher unterscheiden, man spricht von \textit{cross-functional} Teams.  Ein Mitglied des Teams (Scrum Master) leitet das Team und vertritt es nach außen. Die Rollen sind daher meist klar, wenn nicht sollten sie klar festgelegt und definiert werden. Ebenso ist bei der Verantwortung: In agilen Teams trägt jedes Teammitglied die Verantwortung selbst, wenn nicht muss auch dies definiert werden. Das Team kann bei Vertrauen, Motivation und selbstorganisation die Verantwortung gut tragen.
\subsection{Problembehandlung}
todo
\section{Herausforderungen in der Entwicklung in Teams (technische Aspekte)}
%%%%%%%%%%%%%%%%%%%%%%%%%%%%%%%%%%%%%%%%%%%%%%%%%%%%%%%%%%%%%%%%%%%%%%%%%%%%%

\chapterend