%%%%%%%%%%%%%%%%%%%%%%%%%%%%%%%%%%%%%%%%%%%%%%%%%%%%%%%%%%%%%%%%%%%%%%%%%%%%%
\chapter{Stand der Technik und Konzepte}
\chapterstart

\label{chap:related}
\section{Herausforderungen in der Entwicklung in Teams (nicht-technische Aspekte)}
\subsection{Agile Teams}
Einer der großen Herausforderungen in der heutigen Softwareentwicklung ist die Entwicklung in Teams. Soziale Herausforderungen, Probleme im Prozess, Schwierigkeiten im Unternehmen und andere Aspekte sind ebenso wichtig in der Entwicklung in Teams, wie die technisches Aspekte. In dieser Arbeit werden Teams als Scrum-Teams im agilen Prozess/Unternehmen betrachtet: Die Teams können sich selbst organisieren und entwickeln sich selbst im Rahmen von Feedback, Erfahrung und Meetings weiter \ref{cohn2003introducing}. Dieses Team arbeitet im Rahmen eines Prinzipes, welches ist \ref{beck2001agile}
\begin{itemize} 
\item Individuen und Interaktionen mehr als Prozesse und Werkzeuge
\item Funktionierende Software mehr als umfassende Dokumentation
\item Zusammenarbeit mit dem Kunden mehr als Vertragsverhandlung
\item Reagieren auf Veränderung mehr als das Befolgen eines Plans
\end{itemize}
\subsection{Teamkultur}
Unter einer Teamkultur versteht man das pflegen von Regeln, Richtlinien und Verhaltensweisen \ref{teamculture}. Eine gute Teamkultur ist die Basis für eine funktionierendes Team. Darunter fallen Bereiche wie, wie gemeine Vorstellungen und Ziele mit dem Projekt, Motivation, Umgang mit Fehlern, ein aufrichtiges Feedback und eine Gleichberechtigung untereinander im Team. \ref{teamcultureGrolman}. 
\subsection{Kommunikation}
\subsection{Verteilte Teammitglieder}
\section{Herausforderungen in der Entwicklung in Teams (technische Aspekte)}
%%%%%%%%%%%%%%%%%%%%%%%%%%%%%%%%%%%%%%%%%%%%%%%%%%%%%%%%%%%%%%%%%%%%%%%%%%%%%

\chapterend