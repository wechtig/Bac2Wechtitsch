\chapter{Einleitung}

\section{Problemstellung}
Code Quality ist ein oft unterschätzter Teil der Softwareentwicklung. 
Viele Bugs und Sicherheitslücken können durch eine gesteigerte Code Quality vermieden werden. Teile der Code Quality sind unter anderem die Wartbarkeit und die Möglichkeiten des Refactorings. Darunter versteht man einerseits den Aufwand um den Code zu verstehen und zu erweitern und andererseits die Verbesserung des Codes, ohne dessen Verhalten zu ändern.
Beide Faktoren beeinflussen die Teamentwicklung stark, da diese Aspekte der Code Qualität bei einer gemeinsamen Entwicklung der Software sehr wichtig sind um Zeit und Aufwand zu sparen.

Herkömmliche Lösungen konzentrieren sich auf die Anzeige der einzelnen Fehler und Bugs, die mit Hilfe von Code-Qualität-Tools herausgefunden werden. Der Fokus liegt auf der Ausbesserung des Fehlers. Auch die Kenntnisse und Ergebnisse aus der Bachelorarbeit 1 stellen den einzelnen Entwickler und die Fehlerbehebung und nicht das Team in den Mittelpunkt.
Ebenso ist es mit großen Projekten nicht mehr nachvollziehbar, welche Person den Fehler implementiert hat. Die Fehler werden alle zusammen in einer großen Auflistung angezeigt. Das verhindert auch eine weitere Verbesserung der Kenntnisse der Entwicklerinnen und Entwickler, da nicht bekannt welche Fehler öfters wiederholt und so vermeiden sollte. 

\section{Zielsetzung und Forschungsfragen}

Dadurch stellt sich die Frage, wie herkömmliche Lösungsansätze in eine Teamfunktionalität und Unterstützung integriert werden können. Das Ziel ist es, Lösungsansätze zu untersuchen und entwickeln, die das Team im Gesamten unterstützen und damit die Code Qualität zu verbessern.  

\subsection{Methodik} 
Die Applikation basiert auf dem Entwicklungsstand der Bachelorarbeit 1:
Mithilfe einer Webapplikation werden Daten verschiedener Code-Analyse-Tools ausgewertet und präsentiert. Dazu werden in mehreren verschiedenen Projekten diese Tools eingesetzt. Die Analysen in der Webapplikation bauen auf diese Daten auf, die mithilfe des Plugins gespeichert werden. 
Ebenso wird eine mobile Applikation entwickelt, die die Daten aus der Webapplikation benötigt. 

Um die Effizienz, Einfachheit bei der Anwendung, Mehrwertigkeit und Unterstützungshilfe der Applikationen und der Hilfsmittel festzustellen, werden Tests und Anwendungsfälle mit verschiedenen Entwicklerinnen und Entwicklern durchgeführt. Die durchführenden Benutzerinnen und Benutzer werden hierbei in Teams aufgeteilt die zusammen ein Projektteam darstellten sollen. Die Personen sollen hierbei einen unterschiedlichen Wissensstand im Bereich der Softwareentwicklung aufweisen, sodass die Ergebnisse nur auf der Webapplikation und nicht auf Wissen über bestimmte Tools und Fehler basieren. 

Beim Durchführen der Tests muss darauf geachtet werden, dass sich das Testsetup und die Testumgebung nicht unterscheidet. Die Ergebnisse der Tests werden protokolliert. Im Test können die Testpersonen mit der Applikation direkt und interaktiv arbeiten. Dies geschieht im Rahmen eines Interviews, wo die Anwenderinnen und Anwender Erfahrungen mit der Applikation, Kritikpunkte und Vorschläge einbringen können. Der Test beginnt mit einer zurückgesetzten Datenbank und einem leeren Frontend. Mittels einer Bildschirmübertragung, bei der die Testpersonen auch die Steuerung des Computers übernehmen können, wird der Test durchgeführt.
Der Test beinhaltet die Beantwortung von vorgefertigten Fragen, das Ausführen der Funktionen, das Suchen von angezeigten Fehlern sowie offenes Feedback. Die Tests finden online statt und haben einen Zeitrahmen von 30 bis 45 Minuten. Der Test für die mobile Anwendung wird in einem Emulator online stattfinden, da so alle Testpersonen die gleiche Basis für den Test haben.

Das Feedback soll sowohl zusammen im Team, als auch alleine geschehen, da sich die Einzelmeinungen nicht beeinflussen sollen.
\subsection{Kriterien} 
Testpersonen evaluieren die Arbeit anhand folgender Kriterien und Punkte:
\begin{itemize}
\item Einfachheit \\ Die Verwendung der Applikation soll unkompliziert und einfach geschehen. 
\item Übersicht \\ Im Frontend der Webapplikation und in der mobilen Lösung sollen alle wichtigen Informationen übersichtlich und gut lesbar aufbereitet werden. Ebenso kann die Benutzerin oder der Benutzer bestimmte Daten selektieren, um einen genaueren Überblick zu bekommen.
\item Unterstützungshilfe \\ Durch verschiedene Funktionalitäten soll die Benutzerin oder der Benutzer eine kurz- und langfristige Unterstützung bei der Entwicklung bekommen.
\item Individualität \\ Die Applikationen soll für die Arbeit der Benutzerinnen und Benutzer und deren eingesetzten Code Analyse Tools verfügbar und kompatibel sein. 
\item Performance \\ Die Benutzerin oder der Benutzer kann nach einsetzen und verwenden der Applikationen schnell seine Daten einsehen.
\item Verständlichkeit \\ Die Funktionalitäten sollen für die Benutzerinnen und Benutzer ohne Hilfestellungen verständlich sein und sollen daher ohne Probleme verwendet und angewandt werden können.
\end{itemize}

Ebenso werden anhand dieser Kriterien die Unterschiede zu herkömmlichen Lösungen erarbeitet. Die Arbeit ist erfolgreich, wenn diese Kriterien zutreffen. 

